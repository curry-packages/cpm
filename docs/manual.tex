\documentclass[11pt]{article}

\usepackage{url}
\usepackage{syntax}
\usepackage{listings}
\usepackage{amsmath}
\lstset{aboveskip=1.5ex,
        belowskip=1.5ex,
        showstringspaces=false,
        mathescape=true,
        flexiblecolumns=false,
        xleftmargin=2ex,
        basewidth=0.52em,
        basicstyle=\small\ttfamily}
\lstset{literate={->}{{$\rightarrow{}\!\!\!$}}3
       }
\renewcommand{\tt}{\usefont{OT1}{cmtt}{m}{n}\selectfont}
\newcommand{\codefont}{\small\tt}
\newcommand{\code}[1]{\mbox{\codefont #1}}
\newcommand{\ccode}[1]{``\code{#1}''}

% The layout of this manual is adapted from the KiCS2 manual.


%%% ------------------------------------------------------------------

\usepackage[colorlinks,linkcolor=blue]{hyperref}
\hypersetup{bookmarksopen=true}
\hypersetup{bookmarksopenlevel=0}
\hypersetup{pdfstartview=FitH}
\usepackage{thumbpdf}

%%% ------------------------------------------------------------------

\setlength{\textwidth}{16.5cm}
\setlength{\textheight}{23cm}
\renewcommand{\baselinestretch}{1.1}
\setlength{\topmargin}{-1cm}
\setlength{\oddsidemargin}{0cm}
\setlength{\evensidemargin}{0cm}
\setlength{\marginparwidth}{0.0cm}
\setlength{\marginparsep}{0.0cm}

\begin{document}

\title{CPM User's Manual}

\author{Jonas Oberschweiber \qquad Michael Hanus\\[1ex]
{\small Institut f\"ur Informatik, CAU Kiel, Germany}\\[1ex]
{\small\texttt{packages@curry-language.org}}
}

\maketitle

\begin{abstract}
This document describes the Curry package manager (CPM), a tool to
distribute and install Curry libraries and manage version dependencies
between these libraries.
\end{abstract}

\section{Installing the Curry Package Manager}

To install and use CPM, a working installation of either the
PAKCS\footnote{\url{https://www.informatik.uni-kiel.de/~pakcs/}}
compiler in version 1.14.1 or greater, or the
KiCS2\footnote{\url{https://www-ps.informatik.uni-kiel.de/kics2/}}
compiler in version 0.5.1 or greater is required. Additionally, CPM requires 
\emph{Git}\footnote{\url{http://www.git-scm.com}},
\emph{curl}\footnote{\url{https://curl.haxx.se}}
and \emph{unzip} to be available on the \code{PATH} during installation and 
operation. You also need to ensure that your Haskell installations reads files
using UTF-8 encoding by default. Haskell uses the system locale charmap for its
default encoding. You can check the current value using 
\code{System.IO.localeEncoding} inside a \code{ghci} session.

To install CPM from the sources, enter the 
root directory of the CPM source distribution.
The main executable \code{curry} of your Curry system must be in your
path (otherwise, you can also specify the root location of your Curry system
by modifying the definition of \code{CURRYROOT} in the \code{Makefile}).
Then type \code{make} to compile CPM which generates
a binary called \code{cpm} in the \code{bin} subdirectory. Put
this binary somewhere on your path.

Afterwards, run \code{cpm update} to pull down a copy of the central package 
index to your system.

\section{Package Basics}
\label{sec:package-basics}

Essentially, a Curry package is nothing more than a directory structure 
containing a \code{package.json} file and a \code{src} directory at its root.
The \code{package.json} file is a JSON file containing package metadata, the 
\code{src} directory contains the Curry modules that make up the package.

We assume familiarity with the JSON file format. A good introduction can be 
found at \url{http://json.org}. The package specification file must contain a 
top-level JSON object with at least the keys \code{name}, \code{author}, 
\code{version}, \code{synopsis} and \code{dependencies}. More possible fields 
are described in Section~\ref{sec:reference}. A package's name may contain any
ASCII alphanumeric character as well as dashes (\code{-}) and underscores
(\code{_}). It must start with an alphanumeric character. The author field is a
free-form field, but the suggested format is either a name (\code{John Doe}),
or a name followed by an email address in angle brackets
(\code{John Doe <john.doe@goldenstate.gov>}). Separate multiple authors with
commas. 

Versions must be specified in the format laid out in the semantic versioning 
standard:\footnote{\url{http://www.semver.org}} each version number consists of 
numeric major, minor and patch versions separated by dot characters as well as
an optional pre-release specifier consisting of ASCII alphanumerics and hyphens,
e.g. \code{1.2.3} and \code{1.2.3-beta5}. Please note that build metadata as
specified in the standard is not supported. 

The synopsis should be a short summary of what the package does. Use the 
\code{description} field for longer form explanations. 

Dependencies are specified as a nested JSON object with package names as keys
and dependency constraints as values. A dependency constraint restricts the 
range of versions of the dependency that a package is compatible to. Constraints
consist of elementary comparisons that can be combined into conjunctions, which
can then be combined into one large disjunction -- essentially a disjunctive
normal form. The supported comparison operators are $<, \leq, >, \geq, =$ and 
$\sim>$. The first four are interpreted according to the rules for comparing
version numbers laid out in the semantic versioning standard. $\sim>$ is called
the \emph{semantic versioning arrow}. It requires that the package version be 
at least as large as its argument, but still within the same minor version, i.e.
$\sim> 1.2.3$ would match $1.2.3$, $1.2.9$ and $1.2.55$, but not $1.2.2$ or
$1.3.0$.

To combine multiple comparisons into a conjunction, separate them by commas, 
e.g. $\geq 2.0.0, < 3.0.0$ would match all versions with major version $2$. 
Note that it would not match \textit{2.1.3-beta5} for example, since pre-release 
versions are only matched if the comparison is explicitly made to a pre-release
version, e.g. $= \text{2.1.3-beta5}$ or $\geq \text{2.1.3-beta2}$.

Conjunctions can be combined into a disjunction via the $||$ characters, e.g.
$\geq 2.0.0, < 3.0.0 || \geq 4.0.0$ would match any version within major version
$2$ and from major version $4$ onwards, but no version within major version $3$.


%%%%%%%%%%%%%%%%%%%%%%%%%%%%%%%%%%%%%%%%%%%%%%%%%%%%%%%%%%%%%%%%%%%%%%%%%%%%%%
\section{Using Packages}

Curry packages can be used as dependencies of other Curry packages
or to install applications implemented with a package.
In the following we describe both possibilities of using packages.

\subsection{Creating New Packages}

Creating a new Curry package is easy.
To use a Curry package in your project, create a 
\code{package.json} file in the root, fill it with the minimum amount of 
information discussed in the previous session, and move your Curry code to a
\code{src} directory inside your project's directory. Alternatively, if you are
starting a new project, use the \code{cpm new <project-name>} command, which 
creates a new project directory with a 
\code{package.json} file for you.\footnote{The \code{new} command
also creates some other useful template files. Look into the
output of this command.}
Declare a dependency inside the new \code{package.json} file, e.g.:

\begin{lstlisting}
{
  ...,
  "dependencies": {
    "json": "~> 1.1.0"
  }
}
\end{lstlisting}
%
Then run \code{cpm install} to install all dependencies of the current package
and start your interactive Curry environment with \code{cpm curry}. You will be
able to load the JSON package's modules in your Curry session.


\subsection{Installing and Updating Dependencies}

To install the current package's dependencies, run \code{cpm install}. This will
install the most recent version of all dependencies that are compatible to the
package's dependency constraints. Note that a subsequent run of 
\code{cpm install} will always prefer the versions it installed on a previous
run, if they are still compatible to the package's dependencies. If you want to
explicitly install the newest compatible version regardless of what was 
installed on previous runs of \code{cpm install}, you can use the 
\code{cpm upgrade} command to upgrade all dependencies to their newest 
compatible versions, or \code{cpm upgrade <package>} to update a specific 
package and all its transitive dependencies to the newest compatible version.

Note that there is also a \code{cpm update} command, which will update
your copy of the central package index to the newest version. You can
list all packages of the central package index via the
\code{cpm list} command, or you can
search the central package index via the \code{cpm search}
command. See Section~\ref{sec:cmd-reference} for a reference of all
commands.

If the package also contains an implementation of a complete executable,
e.g., some useful tool,
which can be specifed in the \code{package.json} file
(see Section~\ref{sec:reference}),
then the command \code{cpm install} also compiles the application
and installs the executable in the \code{bin} install directory of CPM
(see Section~\ref{sec:config} for details).
The installation of executables can be suppressed by the
\code{cpm install} option \code{-n} or \code{--noexec}.


\subsection{Checking out Packages}
\label{sec:checkout}

In order to use, experiment with or modify an existing package,
one can use the command
\begin{lstlisting}
cpm checkout <package>
\end{lstlisting}
to install a local copy of a package.
This is also useful to install some tool distributed as a package.
For instance, to install \code{curry-genmake},
a tool to generate a \code{make} file for a Curry application,
one can check out the most recent version and install the tool:
%
\begin{lstlisting}
> cpm checkout makefile
$\ldots$ Package 'makefile-1.3.4' checked out into directory 'makefile'.
> cd makefile
> cpm install
$\ldots$
INFO  Installing executable 'curry-genmake' into '/home/joe/.cpm/bin'
\end{lstlisting}
%
Now, the tool \code{curry-genmake} is ready to use
if \code{\$HOME/.cpm/bin} is in your path
(see Section~\ref{sec:config} for details about changing the location
of this default path).


\subsection{Installing Applications of Packages}
\label{sec:installapp}

Some packages do not contain only useful libraries
but also application programs or tools.
In order to install the executables of such applications without
explicitly using the source code of the package,
one can use the command
\begin{lstlisting}
cpm installapp <package>
\end{lstlisting}
This command checks out the package in some internal directory
(default: \code{\$HOME/.cpm/app_packages}, see
Section~\ref{sec:config})
and installs the binary of the application provided by the package
in \code{\$HOME/.cpm/bin} (see also Section~\ref{sec:checkout}).

For instance, the most recent version of the web framework Spicey
can be installed by the following command:
%
\begin{lstlisting}
> cpm installapp spicey
$\ldots$ Package 'spicey-xxx' checked out $\ldots$
$\ldots$
INFO  Installing executable 'spiceup' into '/home/joe/.cpm/bin'
\end{lstlisting}
%
Now, the binary \code{spiceup} of Spicey can be used
if \code{\$HOME/.cpm/bin} is in your path
(see Section~\ref{sec:config} for details about changing the location
of this default path).


\subsection{Executing the Curry Compiler}

To use the dependencies of a package, the Curry compiler needs to be
started via CPM so that the compiler will know where to search for the
modules provided. You can use the command \ccode{cpm curry} to start the
Curry compiler (which is either the compiler used to install CPM
or specified with the configuration option \code{CURRY_BIN},
see Section~\ref{sec:config}).
Any parameters given to \ccode{cpm curry} will be passed along verbatim to
the Curry compiler.
For example, the following will start the Curry
compiler, print the result of evaluating the expression \code{39+3}
and then quit.

\begin{lstlisting}
> cpm curry :eval "39+3" :quit
\end{lstlisting}
%
To execute other Curry commands, such as \ccode{curry check},
with the package's dependencies available,
you can use the \ccode{cpm exec} command.
This command will set the \code{CURRYPATH} environment variable
and then execute the command given after \ccode{exec}.


\subsection{Replacing Dependencies with Local Versions}
\label{sec:cpm-link}

During development of your applications, situations may arise in which you want
to temporarily replace one of your package's dependencies with a local copy,
without having to publish a copy of that dependency somewhere or increasing the
dependency's version number. One such situation is a bug in a dependency not 
controlled by you: if your own package depends on package $A$ and $A$'s current
version is $1.0.3$ and you encounter a bug in this version, then you might be 
able to investigate, find and fix the bug. Since you are not the the author of
$A$, however, you cannot release a new version with the bug fixed. So you send
off your patch to $A$'s maintainer and wait for $1.0.4$ to be released. In the
meantime, you want to use your local, fixed copy of version $1.0.3$ from your 
package. The \code{cpm link} command allows you to replace a dependency with
your own local copy.

\code{cpm link} takes a directory containing a copy of one of the current 
package's dependencies as its argument. It creates a symbolic link from that
directory the the current package's local package cache. If you had a copy of
\code{A-1.0.3} in the \code{~/src/A-1.0.3} directory, you could use 
\code{cpm link ~/src/A-1.0.3} to ensure that any time \code{A-1.0.3} is used 
from the current package, your local copy is used instead of the one from the
global package cache. To remove any links, use \code{cpm upgrade} without any
arguments, which will clear the local package cache. See 
Section~\ref{sec:internals} for more information on the global and local package
caches.

\section{Authoring Packages}

If you want to create packages for other people to use, you should consider 
filling out more metadata fields than the bare minimum. See 
Section~\ref{sec:reference} for a reference of all available fields.

\subsection{Semantic Versioning}
\label{sec:semantic-versioning}

The versions of published packages should adhere to the semantic versioning 
standard, which lays out rules for which components of a version number must
change if the public API of a package changes. Recall that a semantic versioning 
version number consists of a major, minor and patch version as well as an 
optional pre-release specifier. In short, semantic versioning defines the 
following rules:

\begin{itemize}
\item If the type of any public API is changed or removed or the expected 
behavior of a public API is changed, you must increase the major version number
and reset the minor and patch version numbers to $0$.

\item If a public API is added, you must increase at least the minor version
number and reset the patch version number to $0$.

\item If only bug fixes are introduced, i.e. nothing is added or removed and
behavior is only changed to removed deviations from the expected behavior, then
it is sufficient to increase the patch version number.

\item Once a version is published, it must not be changed.

\item For pre-releases, sticking to these rules is encouraged but not required.

\item If the major version number is $0$, the package is still considered under
development and thus unstable. In this case, the rules do not apply, although
following them as much as possible as still encouraged. Release $1.0.0$ is 
considered to be the first stable version.
\end{itemize}
%
To aid you in following these rules, CPM provides the \code{diff} command.
\code{diff} can be used to compare the types and behavior of a package's public
API between two versions of that package. If it finds any differences, it checks
whether they are acceptable under semantic versioning for the difference in 
version numbers between the two package versions. To use \code{diff}, you need
to be in the directory of one of the versions, i.e., your copy for development,
and have the other version installed in CPM's global package cache (see the
\code{cpm install} command). For example, if you are developing version $1.3.0$
of the JSON package and want to make sure you have not introduced any breaking
changes when compared to the previous version $1.2.6$, you can use the 
\code{cpm diff 1.2.6} command while in the directory of version $1.3.0$. 

CPM will then check the types of all public functions and data types in all
exported modules of both versions (see the \code{exportedModules} field of the
package specification) and report any differences and whether they violate 
semantic versioning. It will also generate a CurryCheck program that will 
compare the behavior of all exported functions in all exported modules whose
types have not changed and execute that program. Note that not all functions
can be compared via CurryCheck.
In particular, functions taking other functions as arguments
(there are a few other minor restrictions)
can not be checked so that CPM automatically excludes them from checking.

Note that the results of non-terminating operations, like \code{Prelude.repeat},
cannot be compared in a finite amount of time.
To avoid the execution of possibly non-terminating check programs,
CPM compares the behavior of operations
only if it can prove the termination or productivity\footnote{%
An operation is productive if it always produces outermost constructors,
i.e., it cannot run forever without producing constructors.}
of these operations.
Since CPM uses simple criteria to approximate these properties,
there might be operations that are terminating or productive
but CPM cannot show it. In these cases you can use the compiler pragmas
\verb|{-# TERMINATE -#}| or \verb|{-# PRODUCTIVE -#}| to annotate such
functions. Then CPM will trust these annotations and treat
the annotated operations as terminating or productive, respectively.
For instance, CPM will check the following operation although
it cannot show its termination:

\begin{lstlisting}
{-# TERMINATE -#}
mcCarthy :: Int -> Int
mcCarthy n | n<=100 = mcCarthy (mcCarthy (n+11))
           | n>100 = n-10
\end{lstlisting}
%
As another example, consider the following operation defining
an infinite list:

\begin{lstlisting}
ones :: [Int]
ones = 1 : ones
\end{lstlisting}
%
Although this operation is not terminating, it is productive
since with every step a new constructor is produced.
CPM compares such operations by comparing their results up to
some depth.
On the other hand, the following operation is not classified
as productive by CPM (note that it would not be productive if the
filter condition is changed to \code{(>1)}):

\begin{lstlisting}
{-# PRODUCTIVE -#}
anotherOnes :: [Int]
anotherOnes = filter (>0) ones
\end{lstlisting}
%
Due to the pragma, CPM will compare this operation as other productive
operations.

There might be situations when operations should not be compared,
e.g., if the previous version of the operation was buggy.
In this case, one can mark those functions with the compiler pragma
\verb|{-# NOCOMPARE -#}| 
so that CPM will not generate tests for them.


\subsection{Publishing a Package}
\label{sec:publishing-a-package}

There are three things that need to be done to publish a package: make the
package accessible somewhere, add the location to the package specification, and
add the package specification to the central package index.

CPM supports ZIP files accessible over HTTP as well as Git repositories as 
package sources. You are free to choose one of those, but a publicly accessible
Git repository is preferred. To add the location to the package specification,
use the \code{source} key. For a HTTP source, use:
%
\begin{lstlisting}
{
  ...,
  "source": {
    "http": "http://example.com/package-1.0.3.zip"
  }
}
\end{lstlisting}
%
For a Git source, you have to specify both the repository as well as the 
revision that represents the version:
%
\begin{lstlisting}
{
  ...,
  "source": {
    "git": "git+ssh://git@github.com:john-doe/package.git",
    "tag": "v1.2.3"
  }
}
\end{lstlisting}
%
There is also a shorthand, \code{\$version}, available to
automatically use a tag consisting of the letter \code{v} followed by
the current version number, as in the example above. Specifying
\code{\$version} as the tag and then tagging each version in the
format \code{v1.2.3} is preferred, since it does not require changing
the source location in the \code{package.json} file every time a new
version is released.
If one already has a repository with another tagging scheme,
one can also place the string \code{\$version\$}
in the tag, which will be automatically replaced by the current
version number. Thus, the tag \ccode{\$version} is equivalent
to the tag \ccode{v\$version\$}.

After you have published the files for your new package version, you
have to add the corresponding package specification to the central
package index. The central package index is just a Git repository
containing a directory for each package, which contain subdirectories
for all versions of that package which in turn contain the package
specification files. So the specification for version $1.0.5$ of the
\code{json} package would be located in
\code{json/1.0.5/package.json}.  If you have access to the Git
repository containing the central package index, then you can add the
package specification yourself.  Otherwise, send your package
specification file to \url{packages@curry-language.org} in order to
publish it.


\section{Configuration}
\label{sec:config}

CPM can be configured via the \code{\$HOME/.cpmrc} configuration file. The 
following list shows all configuration options and their default values.

\begin{description}
\item[\fbox{\code{REPOSITORY_PATH}}] The path to the index repository.
Default value: \code{\$HOME/.cpm/index}.

\item[\fbox{\code{PACKAGE_INSTALL_PATH}}] The path to the global package cache.
This is where all downloaded packages are stored.
Default value: \code{\$HOME/.cpm/packages}

\item[\fbox{\code{BIN_INSTALL_PATH}}] The path to the executables
of packages. This is the location where the compiled executables
of packages containing full applications are stored.
Hence, in order to use such applications, one should have this path
in the personal load path (environment variable \code{PATH}).
Default value: \code{\$HOME/.cpm/bin}

\item[\fbox{\code{APP_PACKAGE_PATH}}]
The path to the package cache where packages are checked out if only
their binaries are installed (see Section~\ref{sec:installapp}).
Default value: \code{\$HOME/.cpm/app_packages}.

\item[\fbox{\code{CURRY_BIN}}]
The name of the executable of the Curry system used
to compile and test packages.
The default value is the binary of the Curry system which has been used
to compile CPM.
\end{description}
%
Note that one write the option names also in lowercase or omit
the underscores. For instance, one can also write \code{currybin}
instead of \code{CURRY_BIN}.
Moreover, one can override the values of these configuration options
by the CPM options \code{-d} or \code{--define}.
For instance, to install the binary
of the package \code{spicey} in the directory \code{\$HOME/bin},
one can execute the command
\begin{lstlisting}
> cpm --define bin_install_path=$\$$HOME/bin installapp spicey
\end{lstlisting}


\section{Some CPM Internals}
\label{sec:internals}

CPM's central package index is a Git repository containing package specification
files. A copy of this Git repository is stored on your local system in the 
\code{\$HOME/.cpm/index} directory, unless you changed the location using the
\code{REPOSITORY_PATH} setting. CPM uses the package index when searching for 
and installing packages and during dependency resolution. 

When a package is installed on the system, it is stored in the
\emph{global package cache}.
By default, the global package cache is located in
\code{\$HOME/.cpm/packages}. When a package \emph{foo}, stored in directory 
\code{foo}, depends on a package \emph{bar}, a link to \emph{bar's} directory in
the global package cache is added to \emph{foo's} local package cache when 
dependencies are resolved for \emph{foo}.
The \emph{local package cache} is stored in 
\code{foo/.cpm/package_cache}. Whenever dependencies are resolved, package 
versions already in the local package cache are preferred over those from the
central package index or the global package cache. 

When a module inside a package is compiled, packages are first copied from the
local package cache to the \emph{run-time cache}, which is stored in 
\code{foo/.cpm/packages}. Ultimately, the Curry compiler only sees the package
copies in the run-time cache, and never those from the local or global package
caches.

\section{Command Reference}
\label{sec:cmd-reference}

This section gives a short description of all available CPM commands. In 
addition to the commands listed here, there is a global parameter
\code{--verbosity}
which defaults to \code{info} but can be increased to \code{debug} for
more output.
Furthermore, there is a global parameter \code{--define} to override
the configuration options of CPM, see Section~\ref{sec:config}.

\begin{description}
\item[\fbox{\code{info}}] Gives information on the current package, e.g. the
package's name, author, synopsis and its dependency specifications.

\item[\fbox{\code{info $package$ [--all]}}]
Prints information on the newest known version (compatible to the
current compiler) of the given package.
The option \code{--all} shows more information.

\item[\fbox{\code{info $package$ $version$ [--all]}}]
Prints basic information on the given package version.
The option \code{--all} shows more information.

\item[\fbox{\code{list [--versions] [--csv]}}]
List the names and synopses of all packages of the central package index.
Unless the option \code{--versions} is set, only the newest version
of a package (compatible to the current compiler) is shown.
The option \code{--versions} shows all versions of the packages.
If a package is not compatible to the current compiler, then
the package version is shown as \ccode{???}.
The option \code{--csv} shows the information in CSV format.

\item[\fbox{\code{list --category [--csv]}}]
List the category names together with the packages belonging to this
category (see Section~\ref{sec:reference})
of the central package index.
The option \code{--csv} shows the information in CSV format.

\item[\fbox{\code{search [--module] $query$}}]
Searches the names, synopses, and exported module names of all 
packages of the central package index for occurrences of the given
search term.
If the option \code{--module} is set, then the search is restricted
to occurrences of an exported module. Thus, the package
exporting the module \code{JSON.Data} can be found by the command
%
\begin{lstlisting}
> cpm search --module JSON.Data
\end{lstlisting}

\item[\fbox{\code{update}}] Updates the local copy of the central package index
to the newest available version.

\item[\fbox{\code{install}}] Installs all dependencies of the current package.
Furthermore, if the current package contains an executable application,
the application is compiled and the executable is installed
(unless the option \code{-n} or \code{--noexec} is set).

\item[\fbox{\code{install $package$ [--$pre$]}}] Installs the newest version
(compatible to the current compiler) of 
a package to the global package cache. \code{--$pre$} enables the installation
of pre-release versions.

\item[\fbox{\code{install $package$ $version$}}] Installs a specific version of
a package to the global package cache.

\item[\fbox{\code{install $package$.zip}}] Installs a package from a ZIP file
to the global package cache. The ZIP file must contain at least the
package description file \code{package.json} and the directory \code{src}
containing the Curry source files.

\item[\fbox{\code{uninstall}}] Uninstall the executable installed
for this package.

\item[\fbox{\code{uninstall $package$ $version$}}] Uninstalls a specific version
of a package from the global package cache.

\item[\fbox{\code{checkout $package$ [--$pre$]}}]
Checks out the newest version (compatible to the current compiler)
of a package into the local directory \code{$package$}
in order to test its operations or install a binary of the package.
\code{--$pre$} enables the installation of pre-release versions.

\item[\fbox{\code{checkout $package$ $version$}}]
Checks out a specific version of a package
into the local directory \code{$package$}
in order to test its operations or install a binary of the package..

\item[\fbox{\code{installapp $package$ [--$pre$]}}]
Install the application provided by the newest version
(compatible to the current compiler) of a package.
The binary of the application is installed into the directory
\code{\$HOME/.cpm/bin}
(this location can be changed via the \code{\$HOME/.cpmrc} configuration file
or by the CPM option \code{--define}, see Section~\ref{sec:config}).
\code{--$pre$} enables the installation of pre-release versions.

\item[\fbox{\code{installapp $package$ $version$}}]
Install the application provided by a specific version of a package.
The binary of the application is installed into the directory
\code{\$HOME/.cpm/bin}
(this location can be changed via the \code{\$HOME/.cpmrc} configuration file
or by the CPM option \code{--define}, see Section~\ref{sec:config}).

\item[\fbox{\code{upgrade}}]
Upgrades all dependencies of the current package to
the newest compatible version.

\item[\fbox{\code{upgrade $package$}}] Upgrades a specific dependency of the
current package and all its transitive dependencies to their newest compatible
versions.

\item[\fbox{\code{deps}}] Does a dependency resolution run for the current 
package and prints out the results. The result is either a list of all package
versions chosen or a description of the conflict encountered during dependency
resolution.

\item[\fbox{\code{test}}]
Tests the current package with CurryCheck.
If the package specification contains a definition of a test suite
(entry \code{testsuite}, see Section~\ref{sec:reference}),
then the modules defined there are tested.
If there is no test suite defined,
the list of exported modules are tested,
if they are explicitly specified
(field \code{exportedModules} of the package specification),
otherwise all modules in the directory \code{src}
(including hierarchical modules stored in its subdirectories) are tested.
Using the option \code{--modules}, one can also specify a comma-separated
list of module names to be tested.

\item[\fbox{\code{doc}}]
Generates the HTML documentation of the current package with CurryDoc.
If the package specification contains a list of exported modules
(see Section~\ref{sec:reference}),
then these modules are documented.
Otherwise, the main module (if the package specification contains
the entry \code{executable}, see Section~\ref{sec:reference})
or all modules in the directory \code{src}
(including hierarchical modules stored in its subdirectories)
are documented.
Using the option \code{--modules}, one can also specify a comma-separated
list of module names to be documented.
Using the option \code{--docdir}, one can specify the
target directory where the documentation should be stored.
If this option is not provided, \ccode{cdoc} is used as the documentation
directory.

\item[\fbox{\code{diff [$version$]}}]
Compares the API and behavior of the current package to another
version of the same package.
If the version option is missing, the latest version of the current package
found in the repository is used for comparison.
If the options \code{--api-only} or \code{--behavior-only} are added,
then only the API or the behavior are compared, respectively.
Using the option \code{--modules}, one can also specify a comma-separated
list of module names to be compared. Without this option,
all exported modules are compared.

As described in Section~\ref{sec:semantic-versioning},
CPM uses property tests to compare the behavior
of different package versions. In order to avoid
infinite loops durings these tests, CPM analyzes the termination
behavior of the involved operations.
Using the operation \code{--unsafe}, CPM omits this program analysis
but then you have to ensure that all operations are terminating
(or you can annotate them by pragmas,
see Section~\ref{sec:semantic-versioning}).

\item[\fbox{\code{exec $command$}}] Executes an arbitrary command with the 
\code{CURRYPATH} environment variable set to the paths of all dependencies of
the current package.
For example, it can be used to execute \ccode{curry check}
or \ccode{curry analyze} with correct dependencies available.

\item[\fbox{\code{curry $args$}}] Executes the Curry compiler with the 
dependencies of the current package available.
Any arguments are passed verbatim to the compiler.

\item[\fbox{\code{link $source$}}] Can be used to replace a dependency of the 
current package using a local copy, see Section~\ref{sec:cpm-link} for details.

\item[\fbox{\code{clean}}] Cleans the current package from the
generated auxiliariy files, e.g., intermediate Curry files,
installed dependent packages, etc.
Note that a binary installed in the CPM \code{bin} directory
(by the \code{install} command) will not be removed.
Hence, this command can be used to clean an application package
after installing the application.

\item[\fbox{\code{new $project$}}]
Creates a new project package with the given name and some template files.
\end{description}


\section{Package Specification Reference}
\label{sec:reference}

This section describes all metadata fields available in a CPM package 
specification. Mandatory fields are marked with a \code{*} character.

\begin{description}
\item[\fbox{\code{name*}}] The name of the package. Must only contain ASCII 
letters, digits, hyphens and underscores. Must start with a letter.

\item[\fbox{\code{version*}}] The version of the package. Must follow the format
for semantic versioning version numbers.

\item[\fbox{\code{author*}}] The package's author. This is a free-form field, 
the suggested format is either a name or a name followed by an email address in
angle brackets, e.g., \code{John Doe <john@doe.com>}. Multiple authors should
be separated by commas.

\item[\fbox{\code{maintainer}}] The current maintainers of the package, if 
different from the original authors. This field allows the current maintainers
to indicate the best person or persons to contact about the package while 
attributing the original authors.

The suggested format is a name followed by an email address in
angle brackets, e.g., \code{John Doe <john@doe.com>}.
Multiple maintainers should be separated by commas.

\item[\fbox{\code{synopsis*}}] A short form summary of the package's purpose.
It should be kept as short as possible (ideally, less than 100 characters).

\item[\fbox{\code{description}}] A longer form description of what the package 
does.

\item[\fbox{\code{category}}]
A list of keywords that characterize the main area where the
package can be used, e.g., \code{Data}, \code{Numeric}, \code{GUI},
\code{Web}, etc.

\item[\fbox{\code{license}}] The license under which the package is distributed.
This is a free-form field. In case of a well-known license such as the GNU 
General Public License\footnote{\url{https://www.gnu.org/licenses/gpl-3.0.en.html}},
the SPDX license identifier\footnote{\url{https://spdx.org/licenses/}} should be 
specified. If a custom license is used, this field should be left blank in favor
of the license file field.

\item[\fbox{\code{licenseFile}}] The name of a file in the root directory of the
package containing explanations regarding the license of the package or the full
text of the license. The suggested name for this file is \code{LICENSE}.

\item[\fbox{\code{copyright}}] Copyright information regarding the package.

\item[\fbox{\code{homepage}}] The package's web site. This field should contain
a valid URL.

\item[\fbox{\code{bugReports}}] A place to report bugs found int he package. The
suggested formats are either a valid URL to a bug tracker or an email address.

\item[\fbox{\code{repository}}] The location of a SCM repository containing the
package's source code. Should be a valid URL to either a repository (e.g. a Git
URL), or a website representing the repository.

\item[\fbox{\code{dependencies*}}] The package's dependencies. This must be JSON
object where the keys are package names and the values are version 
constraints. See Section~\ref{sec:package-basics}.

\item[\fbox{\code{compilerCompatibility}}] The package's compatibility to 
different Curry compilers. Expects a JSON object where the keys are compiler 
names and the values are version constraints. Currently, the supported compiler
names are \code{pakcs} and \code{kics2}. If this field is missing or contains
an empty JSON object, the package is assumed to be compatible to all compilers
in all versions.

The compiler compatibility of a package is also relevant when
some version of a package should be examined or installed
(with CPM commands \code{info}, \code{checkout}, \code{install},
\code{installapp}).
If a newest package should be installed, i.e., no specific version
number is provided, then only the newest version
which is compatible to the current Curry compiler
(see also Section~\ref{sec:config} for configuration option \code{CURRY_BIN})
is considered.
Similarly, the current package is executed
(CPM commands \code{curry} and \code{test})
only if the current Curry compiler is compatible to this package.

\item[\fbox{\code{source}}] A JSON object specifying where the version of the
package described in the specification can be obtained. See 
Section~\ref{sec:publishing-a-package} for details.

\item[\fbox{\code{sourceDirs}}] A list of directories inside this
package where the source code is located.
When the package is compiled, these directories are put at the front
of the Curry load path.
If this field is not specified, \code{src} is used as the single
source directory.

\item[\fbox{\code{exportedModules}}] A list of modules intended for use by 
consumers of the package.
These are the modules compared by the \code{cpm diff}
command (and tested by the \code{cpm test} command if a list of
test modules is not provided).
Note that modules not in this list are still accessible to consumers
of the package.

\item[\fbox{\code{configModule}}]
A module name into which some information about the package configuration
(location of the package directory, name of the executable, see below)
is written when the package is installed.
This could be useful if the package needs some data files
stored in this package during run time.
For instance, a possible specification could be as follows:
%
\begin{lstlisting}
{
  ...,
  "configModule": "CPM.PackageConfig",
  ...
}
\end{lstlisting}
%
In this case, the package configuration is written into the Curry
file \code{src/CPM/PackageConfig.curry}.

\item[\fbox{\code{executable}}]
A JSON object specifying the name of the executable and the main module
if this package contains also an executable application.
The name of the executable must be defined (with key \code{name})
whereas the name of the main module (key \code{main}) is optional.
If the latter is missing, CPM assumes that the main module is \code{Main}.
For instance, a possible specification could be as follows:
%
\begin{lstlisting}
{
  ...,
  "executable": {
    "name": "cpm",
    "main": "CPM.Main"
  }
}
\end{lstlisting}
%
If a package contains an \code{executable} specification,
the command \code{cpm install} also compiles the main module
and installs the executable in the \code{bin} install directory of CPM
(see Section~\ref{sec:config} for details).

\item[\fbox{\code{testsuite}}]
A JSON object specifying a test suite for this package.
This object contains a directory (with key \code{src-dir})
in which the tests are executed.
Furthermore, the test suite must also define a list of
modules to be tested (with key \code{modules}).
For instance, a possible test suite specification could be as follows:
%
\begin{lstlisting}
{
  ...,
  "testsuite": {
    "src-dir": "test",
    "modules": [ "testDataConversion", "testIO" ]
  }
}
\end{lstlisting}
%
All these modules are tested with CurryCheck
by the command \code{cpm test}.
If no test suite is defined, all (exported) modules are tested
in the directory \code{src}.
A test suite can also contain a field \code{options}
which defines a string of options passed to the call to CurryCheck.

If a test suite contains a specific test script instead
modules to be tested with CurryCheck, then one can specify the name
of this test script in the field \code{script}.
In this case, this script is executed in the test directory
(with the possible \code{options} value added).
The script should return the exit code \code{0} if the test is
successful, otherwise a non-zero exit code.
Note that one has to specify either a (non-empty) list of modules
or a test script name in a test suite, but not both.

One can also specify several test suites for a package.
In this case, the \code{testsuite} value is an array
of JSON objects as described above.
For instance, a test suite specification for tests in the
directories \code{test} and \code{examples} could be as follows:
%
\begin{lstlisting}
{
  ...,
  "testsuite": [
     { "src-dir": "test",
       "options": "-v",
       "script": [ "test.sh" ]
     },
     { "src-dir": "examples",
       "options": "-m80",
       "modules": [ "Nats", "Ints" ]
     }
  ]
}
\end{lstlisting}

\end{description}

\section{Error Recovery}
\label{sec:recovery}

There might occur situations when your package or repository
is in an inconsistent state, e.g., when you manually changed
some internal files or such files have been inadvertently changed or
deleted, or a package is broken due to an incomplete download.
Since CPM checks these files, CPM might exit with an error message
that something is wrong.
In such cases, it might be a good idea to clean up your package file system.
Here are some suggestions how to do this:
\begin{description}
\item[\code{cpm clean}]~\\
This command cleans the current package from
generated auxiliariy files (see Section~\ref{sec:cmd-reference}).
Then you can re-install the package and packages on which it depends
by the command \code{cpm install}.
\item[\code{rm -rf \$HOME/.cpm/packages}] ~\\
This cleans all packages which have been previously installed
in the global package cache (see Section~\ref{sec:internals}).
Such an action might be reasonable in case of some download failure.
After clearing the global package cache, all necessary packages
are downloaded again when they are needed.
\item[\code{rm -rf \$HOME/.cpm/index}] ~\\
This removes the central package index of CPM
(see Section~\ref{sec:internals}).
You can simply re-install the newest version of this index
by the command \code{cpm update}.
\end{description}


\end{document}
